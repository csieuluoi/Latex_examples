\documentclass[tikz]{standalone}
\usepackage{ifthen}
\usetikzlibrary{matrix,calc}
\usepackage{pgfplots}

\definecolor{seabornBlue}{RGB}{76,114,176}
\definecolor{seabornRed}{RGB}{196,78,82}
\definecolor{blueBelizeHole}{RGB}{41,128,185}

\pgfkeys{/pgf/number format/.cd, fixed, fixed zerofill, precision=2}

\usepackage{xfp}
\usepgflibrary{fpu}

\newcommand\pgfmathparseFPU[1]{\begingroup%
\pgfkeys{/pgf/fpu,/pgf/fpu/output format=fixed}%
\pgfmathparse{#1}%
\pgfmathsmuggle\pgfmathresult\endgroup}%



\begin{document}

% maximum value is 16384, if greater, this won't work beause of the floating point number.  

%The matrix in numbers
%Horizontal target class
%Vertical output class
\def\myConfMat{{
{16000,  30,  60,   0,  0, 0, 0, 0},  %row 1
{   0,1390, 110,  80,100, 0, 0, 0},  %row 2
{   0,  40,1090,  40,  0, 0, 0, 0},  %row 3
{   0, 350,  30,1020, 90, 0, 0, 0},  %row 4
{   0,  50,  30, 300,800, 0, 0, 0},
{   0,  50,  30, 300,800, 0, 0, 0},
{   0,  50,  30, 300,800, 0, 0, 1},
{   0,  50,  30, 300,800, 0, 100, 10},  %row 5
}}


\def\classNames{{"A","B","C","D","E","F", "G", "H"}} %class names. Adapt at will

\def\numClasses{8} %number of classes. Could be automatic, but you can change it for tests.

%\def\myConfMat{{
%{42856,24,21,7,6,0,2,0},
%{22,1513,16,0,0,0,0,0},
%{36,15,2556,0,0,0,0,0},
%{10,0,0,1570,0,0,0,0},
%{8,0,0,2,4072,0,0,0},
%{0,0,0,0,0,980,0,0},
%{21,0,0,0,0,0,414,0},
%{10,0,0,0,0,6,0,759},
%}}
%
%\def\classNames{{"Normal(0)", "NMRI(1)", "CMRI(2)", "MSCI(3)", "MPCI(4)", "MFCI(5)", "DoS(6)", "Recon(7)"}} %class names. Adapt at will
%
%\def\numClasses{8} %number of classes. Could be automatic, but you can change it for tests.

\def\myScale{1.5} % 1.5 is a good scale. Values under 1 may need smaller fonts!
\begin{tikzpicture}[
    scale = \myScale,
    %font={\scriptsize}, %for smaller scales, even \tiny may be useful
    ]

\tikzset{vertical label/.style={rotate=90,anchor=east}}   % usable styles for below
\tikzset{diagonal label/.style={rotate=45,anchor=north east}}


\foreach \y in {1,...,\numClasses} %loop vertical starting on top
{
    % Add class name on the left

    \node [anchor=east, rotate=45] at (0.4,-\y) {\pgfmathparse{\classNames[\y-1]}\pgfmathresult}; 

    \foreach \x in {1,...,\numClasses}  %loop horizontal starting on left
    {
%---- Start of automatic calculation of totSamples for the column ------------   
    \def\totSamples{0}
    \def\d{0.12}
    \foreach \ll in {1,...,\numClasses}
    {
        \pgfmathparse{\myConfMat[\y-1][\ll-1]}%fetch next element
        \xdef\totSamples{\totSamples+\pgfmathresult} %accumulate it with previous sum
        %must use \xdef fro global effect otherwise lost in foreach loop!

    }
    \pgfmathparseFPU{\totSamples} \xdef\totSamples{\pgfmathresult}  % put the final sum in variable

%---- End of automatic calculation of totSamples ----------------
    
    \begin{scope}[shift={(\x,-\y)}]
        \def\mVal{\myConfMat[\y-1][\x-1]} % The value at index y,x (-1 because of zero indexing)
        \pgfmathtruncatemacro{\r}{\mVal}   %
        \pgfmathtruncatemacro{\p}{round(\r/\totSamples*100)}
        
        %% for floating point return from macro: use \pgfmathsetmacro
	    \pgfmathsetmacro{\q}{\r/\totSamples*100}

        \coordinate (C) at (0,0);
        \ifthenelse{\p<50}{\def\txtcol{black}}{\def\txtcol{white}} %decide text color for contrast
        \node[
            draw,                 %draw lines
            text=\txtcol,         %text color (automatic for better contrast)
            align=center,         %align text inside cells (also for wrapping)
            fill=seabornRed!\p,        %intensity of fill (can change base color)
            minimum size=\myScale*10mm,    %cell size to fit the scale and integer dimensions (in cm)
            inner sep=0,          %remove all inner gaps to save space in small scales
            ] (C){\r\\\pgfmathprintnumber{\q}\%};     %text to put in cell (adapt at will)
        %Now if last vertical class add its label at the bottom

        \ifthenelse{\y=\numClasses}{
        \node [rotate = -45] at ($(C)-(0,0.75)$) % can use vertical or diagonal label as option
        {\pgfmathparse{\classNames[\x-1]}\pgfmathresult};}{}
    \end{scope}
    }
}
%Now add x and y labels on suitable coordinates
\coordinate (yaxis) at (-0.3,0.5-\numClasses/2);  %must adapt if class labels are wider!
\coordinate (xaxis) at (0.5+\numClasses/2, -\numClasses-1.25); %id. for non horizontal labels!
\node [vertical label] at (yaxis) {True label};
\node []               at (xaxis) {Predicted label};
\end{tikzpicture}

\end{document}

